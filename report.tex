\documentclass{article}
\usepackage{graphicx}
\usepackage{hyperref}
\usepackage{geometry}
\usepackage{float}
\geometry{a4paper, margin=1in}
\newcommand{\hide}[1]{}

\begin{document}

\begin{titlepage}
    \centering
    
    % Logo 
    \includegraphics[width=0.22\textwidth]{images/scms-logo.png}\par\vspace{0.5cm}
    {\large Department of Scientific Computing, Modeling \& Simulation\par}
    \vspace{1cm}
    
    % Project Title
    {\huge\bfseries Mapping Global Biodiversity:\\[0.2cm]
    Patterns, Trends, and Insights from GBIF Data\par}
    \vspace{2.2cm}
    
    % Submitted By
    {\Large\bfseries Submitted By\par}
    \vspace{0.3cm}
    {\large Amit D. Dhumal - MT2409\par}
    \vspace{1.8cm}
    
    % Submitted To
    {\Large\bfseries Submitted To\par}
    \vspace{0.3cm}
    {\large Dr. Bhalchandra Pujari\par}
    \vspace{2.2cm}

    % Logo 
    \includegraphics[width=0.22\textwidth]{images/PU Logo.png}\par\vspace{0.2cm}
    {\large\bfseries Savitribai Phule Pune University\par}
    \vspace{0.3cm}
       
    \vfill
\end{titlepage}


%\maketitle

\section{Abstract}
This report presents an exploratory analysis of global biodiversity data sourced from the Global Biodiversity Information Facility (GBIF). Biodiversity records offer a window into the living fabric of our planet—yet they often arrive fragmented, uneven, and shaped by where people choose to look. In this project, we trace patterns hidden within nearly 100,000 global observations gathered from the GBIF platform. After carefully cleaning and organizing the data, we explore how life is distributed across kingdoms, continents, and decades. We uncover clear hotspots of reporting activity, seasonal and long-term shifts in observation behavior, and striking differences in species richness across countries. To better understand these patterns, we apply tools such as diversity indices, spatial clustering, and trend forecasting. An interactive dashboard brings these insights to life, allowing users to zoom into regions, species, and time periods of interest. Together, this analysis not only reveals the story told by existing biodiversity data, but also exposes the gaps and biases that shape our understanding of the natural world.


\section{Introduction}
Biodiversity data is crucial for understanding the distribution of life on Earth and monitoring changes over time. Biodiversity data plays a critical role in understanding the distribution, abundance, and dynamics of life on Earth. It serves as the foundation for ecological research, conservation planning, species monitoring, and the assessment of environmental change. The Global Biodiversity Information Facility (GBIF) is one of the largest open-access biodiversity data repositories, integrating millions of species occurrence records contributed by researchers, institutions, and citizen scientists worldwide. These records capture essential information such as species identity, geographic location, sampling method, and observation time, offering a valuable lens through which global biodiversity patterns can be examined.

However, biodiversity data is inherently complex and often marked by challenges such as geographic sampling bias, incomplete taxonomic information, and temporal inconsistencies. Analyzing such data requires careful preprocessing and thoughtful interpretation to ensure meaningful outcomes. In this project, we work with a curated subset of GBIF data—approximately 100,000 occurrence records—to explore large-scale patterns across taxonomic groups, geographic regions, and time periods. The objective is to clean, analyze, and visualize this dataset to uncover insights about species distribution, biodiversity hotspots, and long-term observation trends. By applying both classical exploratory methods and advanced statistical techniques, the study aims to reveal hidden ecological structures while demonstrating the potential and limitations of biodiversity datasets in contemporary environmental research.

\section{Methodology}
\subsection{Data Acquisition and Cleaning}
The dataset (\texttt{dataset\_9.csv}) was loaded and inspected. Key cleaning steps included:
\begin{itemize}
    \item Removing records with missing or invalid coordinates.
    \item Parsing dates to standard datetime format.
    \item Handling missing values in taxonomic fields by labeling them as 'Unknown'.
\end{itemize}
The final cleaned dataset contains approximately 99,763 records.

\subsection{Exploratory Data Analysis (EDA)}
We performed EDA to understand the data structure. Key analyses included:
\begin{itemize}
    \item Taxonomic distribution (Kingdom, Class).
    \item Geographical distribution (Top countries).
    \item Temporal analysis (Yearly and monthly trends).
\end{itemize}

\section{Results and Discussion}

\subsection{Taxonomic Distribution}
The analysis reveals the dominance of certain kingdoms in the dataset.
\begin{figure}[H]
    \centering
    \includegraphics[width=0.9\textwidth]{images/kingdom_distribution.png}
    \caption{Distribution of observations by Kingdom.}
    \label{fig:kingdom}
\end{figure}

\subsection{Geographical Distribution}
Observations are not uniformly distributed globally. Certain countries show higher observation counts, likely due to observer bias or data contribution rates.
\begin{figure}[H]
    \centering
    \includegraphics[width=0.9\textwidth]{images/top_countries.png}
    \caption{Top 15 countries by observation count.}
    \label{fig:countries}
\end{figure}

\newpage

\subsection{Temporal Trends}
The number of observations has changed over time.
\begin{figure}[H]
    \centering
    \includegraphics[width=0.8\textwidth]{images/yearly_trend.png}
    \caption{Yearly trend of observations (post-1900).}
    \label{fig:temporal}
\end{figure}

\section{Data Stories and Interpretation}
This section interprets the patterns observed in the data, focusing on specific scientific narratives.

\subsection{The Latitudinal Diversity Gradient}
One of the most widely recognized patterns in ecology is the Latitudinal Diversity Gradient (LDG), which posits that species richness increases from the poles towards the tropics.
\begin{figure}[H]
    \centering
    \includegraphics[width=0.8\textwidth]{images/latitude_distribution.png}
    \caption{Distribution of observations by latitude.}
    \label{fig:latitude}
\end{figure}
As shown in Figure \ref{fig:latitude}, our data supports this hypothesis to an extent, with a significant concentration of observations in the northern hemisphere's mid-latitudes. However, the drop near the equator suggests a potential sampling bias, where tropical regions, despite being biodiversity hotspots, are under-sampled compared to wealthier northern nations.

\newpage
\subsection{Biodiversity Monitoring in the COVID-19 Era}
The COVID-19 pandemic disrupted human activities globally. We analyzed observation trends from 2015 to 2023 to understand its impact on biodiversity monitoring.
\begin{figure}[H]
    \centering
    \includegraphics[width=0.8\textwidth]{images/covid_trend.png}
    \caption{Observation trends during the COVID-19 era (2015-2023).}
    \label{fig:covid}
\end{figure}
Figure \ref{fig:covid} illustrates the trend. Interestingly, instead of a sharp decline in 2020, we observe a sustained or even increased level of observations. This could be attributed to the rise of "backyard science," where citizen scientists contributed data from their local surroundings during lockdowns.

\subsection{Identifying Under-sampled Regions}
A critical aspect of GBIF data is geographical bias. Figure \ref{fig:countries} highlights that the top contributing countries are predominantly in North America and Europe. This "Western bias" means that vast areas of Africa, South America, and Asia—regions with high endemic biodiversity—are under-represented. This gap presents a significant challenge for global conservation planning.

\section{Dashboard and Advanced Extensions}
An interactive dashboard was developed to facilitate dynamic exploration of the biodiversity data. Initially prototyped, the application was migrated to the **Streamlit** framework to enable rapid development and interactivity.

\subsection{Key Features}
The dashboard integrates several advanced extensions to enhance user experience and analytical depth:
\begin{itemize}
    \item **Streamlit Architecture**: The application utilizes Streamlit's reactive programming model to create a responsive and interactive user interface with minimal boilerplate code.
    \item **Sidebar Navigation**: A sidebar allows users to seamlessly switch between the main "Dashboard" view and the "Advanced Analysis" section.
    \item **Species Search**: A dedicated search bar allows users to query the dataset by \texttt{scientificName}. This feature enables researchers to isolate and study the distribution of specific species of interest.
    \item **Granular Filtering**: Users can filter observations by:
    \begin{itemize}
        \item \textbf{Time Period}: A slider to select specific year ranges.
        \item \textbf{Taxonomy}: Multi-select options for Kingdoms.
        \item \textbf{Geography}: Filtering by specific Country Codes.
    \end{itemize}
\end{itemize}

\subsection{Interactive Visualizations}
To provide actionable insights, the dashboard employs interactive plotting libraries:
\begin{itemize}
    \item **Folium Maps**: A dynamic map visualizes the geospatial distribution of observations. To maintain performance with large datasets, the map intelligently samples up to 10,000 points when the dataset is too large.
    \item **Plotly Charts**: Interactive pie charts and line graphs display taxonomic breakdown and temporal trends, respectively. These charts automatically update based on the applied filters.
    \item **Advanced Analysis Tabs**: The advanced analysis features (Species Richness, Clustering, Trend Forecasting) are organized into tabs for easy access and comparison.
\end{itemize}

\section{Advanced Statistical Analysis}
To deepen our understanding of the biodiversity data, we applied advanced statistical techniques including species richness estimation, spatial clustering, and predictive modeling.

\subsection{Species Richness Estimation}
We calculated the **Shannon Diversity Index** ($H'$) for the top 5 contributing countries to estimate species richness and evenness. The Shannon index provides a more robust measure of biodiversity than simple species counts.
\begin{figure}[H]
    \centering
    \includegraphics[width=0.8\textwidth]{images/shannon_index.png}
    \caption{Shannon Diversity Index for top 5 countries.}
    \label{fig:shannon}
\end{figure}

As shown in Figure \ref{fig:shannon}, the United States exhibits the highest diversity ($H' \approx 8.38$), followed closely by Canada and Australia. This metric helps identify regions with complex ecological communities.
\newpage

\subsection{Cluster Analysis}
We employed **K-Means Clustering** ($k=5$) on the geographical coordinates (latitude and longitude) to identify spatial hotspots of biodiversity observations.
\begin{figure}[H]
    \centering
    \includegraphics[width=0.8\textwidth]{images/geographical_clusters.png}
    \caption{Geographical clusters of observations identified by K-Means.}
    \label{fig:clusters}
\end{figure}

The analysis identified distinct clusters corresponding to major continental regions (North America, Europe, Australia, etc.), validating the global spread of the dataset while highlighting the density of observations in specific zones.

\subsection{Predictive Modeling}
To forecast future observation trends, we implemented a **Random Forest Regression** model on the yearly observation counts (post-1950). This machine learning approach captures non-linear trends more effectively than simple linear models.
\begin{figure}[H]
    \centering
    \includegraphics[width=0.8\textwidth]{images/observation_trend_forecast.png}
    \caption{Random Forest forecast of observation counts.}
    \label{fig:predictive}
\end{figure}

The model achieved an exceptional $R^2$ score of 0.9966, indicating a near-perfect fit to the historical data. Unlike the linear model, the Random Forest model predicts a much more dynamic growth. We project the annual observation count to reach approximately 18,489 by the year 2030, reflecting the accelerating interest in biodiversity monitoring.

\section{Climate Data Integration}
A key innovation in this project was the integration of external climate data.
\subsection{Methodology}
We utilized the \textbf{Open-Meteo API} to fetch historical weather data. To respect API limits and ensure performance, we aggregated biodiversity data by Country and Year, calculating the centroid for each group. We then retrieved:
\begin{itemize}
    \item Mean Annual Temperature ($^\circ$C)
    \item Total Annual Precipitation (mm)
\end{itemize}

\subsection{Correlation Results}
We analyzed the correlation between these climate variables and Species Richness:
\begin{itemize}
    \item \textbf{Temperature}: A weak positive correlation ($\approx 0.11$) was observed.
    \item \textbf{Precipitation}: A moderate positive correlation ($\approx 0.32$) suggests that regions with higher rainfall tend to support a wider variety of species in this dataset.
\end{itemize}

\begin{figure}[H]
    \centering
    \includegraphics[width=0.45\textwidth]{images/climate_correlation_temp.png}
    \includegraphics[width=0.45\textwidth]{images/climate_correlation_precip.png}
    \caption{Correlation of Species Richness with Temperature (Left) and Precipitation (Right).}
\end{figure}



\section{Conclusion}
This project successfully demonstrated the workflow of handling biodiversity data. From cleaning to visualization, we gained insights into the GBIF dataset. The interactive dashboard provides a powerful tool for further exploration. The integration of climate data has further enriched our understanding, highlighting the complex relationship between environmental factors and species richness.

\end{document}
